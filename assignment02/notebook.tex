
% Default to the notebook output style

    


% Inherit from the specified cell style.




    
\documentclass[11pt]{article}

    
    
    \usepackage[T1]{fontenc}
    % Nicer default font (+ math font) than Computer Modern for most use cases
    \usepackage{mathpazo}

    % Basic figure setup, for now with no caption control since it's done
    % automatically by Pandoc (which extracts ![](path) syntax from Markdown).
    \usepackage{graphicx}
    % We will generate all images so they have a width \maxwidth. This means
    % that they will get their normal width if they fit onto the page, but
    % are scaled down if they would overflow the margins.
    \makeatletter
    \def\maxwidth{\ifdim\Gin@nat@width>\linewidth\linewidth
    \else\Gin@nat@width\fi}
    \makeatother
    \let\Oldincludegraphics\includegraphics
    % Set max figure width to be 80% of text width, for now hardcoded.
    \renewcommand{\includegraphics}[1]{\Oldincludegraphics[width=.8\maxwidth]{#1}}
    % Ensure that by default, figures have no caption (until we provide a
    % proper Figure object with a Caption API and a way to capture that
    % in the conversion process - todo).
    \usepackage{caption}
    \DeclareCaptionLabelFormat{nolabel}{}
    \captionsetup{labelformat=nolabel}

    \usepackage{adjustbox} % Used to constrain images to a maximum size 
    \usepackage{xcolor} % Allow colors to be defined
    \usepackage{enumerate} % Needed for markdown enumerations to work
    \usepackage{geometry} % Used to adjust the document margins
    \usepackage{amsmath} % Equations
    \usepackage{amssymb} % Equations
    \usepackage{textcomp} % defines textquotesingle
    % Hack from http://tex.stackexchange.com/a/47451/13684:
    \AtBeginDocument{%
        \def\PYZsq{\textquotesingle}% Upright quotes in Pygmentized code
    }
    \usepackage{upquote} % Upright quotes for verbatim code
    \usepackage{eurosym} % defines \euro
    \usepackage[mathletters]{ucs} % Extended unicode (utf-8) support
    \usepackage[utf8x]{inputenc} % Allow utf-8 characters in the tex document
    \usepackage{fancyvrb} % verbatim replacement that allows latex
    \usepackage{grffile} % extends the file name processing of package graphics 
                         % to support a larger range 
    % The hyperref package gives us a pdf with properly built
    % internal navigation ('pdf bookmarks' for the table of contents,
    % internal cross-reference links, web links for URLs, etc.)
    \usepackage{hyperref}
    \usepackage{longtable} % longtable support required by pandoc >1.10
    \usepackage{booktabs}  % table support for pandoc > 1.12.2
    \usepackage[inline]{enumitem} % IRkernel/repr support (it uses the enumerate* environment)
    \usepackage[normalem]{ulem} % ulem is needed to support strikethroughs (\sout)
                                % normalem makes italics be italics, not underlines
    

    
    
    % Colors for the hyperref package
    \definecolor{urlcolor}{rgb}{0,.145,.698}
    \definecolor{linkcolor}{rgb}{.71,0.21,0.01}
    \definecolor{citecolor}{rgb}{.12,.54,.11}

    % ANSI colors
    \definecolor{ansi-black}{HTML}{3E424D}
    \definecolor{ansi-black-intense}{HTML}{282C36}
    \definecolor{ansi-red}{HTML}{E75C58}
    \definecolor{ansi-red-intense}{HTML}{B22B31}
    \definecolor{ansi-green}{HTML}{00A250}
    \definecolor{ansi-green-intense}{HTML}{007427}
    \definecolor{ansi-yellow}{HTML}{DDB62B}
    \definecolor{ansi-yellow-intense}{HTML}{B27D12}
    \definecolor{ansi-blue}{HTML}{208FFB}
    \definecolor{ansi-blue-intense}{HTML}{0065CA}
    \definecolor{ansi-magenta}{HTML}{D160C4}
    \definecolor{ansi-magenta-intense}{HTML}{A03196}
    \definecolor{ansi-cyan}{HTML}{60C6C8}
    \definecolor{ansi-cyan-intense}{HTML}{258F8F}
    \definecolor{ansi-white}{HTML}{C5C1B4}
    \definecolor{ansi-white-intense}{HTML}{A1A6B2}

    % commands and environments needed by pandoc snippets
    % extracted from the output of `pandoc -s`
    \providecommand{\tightlist}{%
      \setlength{\itemsep}{0pt}\setlength{\parskip}{0pt}}
    \DefineVerbatimEnvironment{Highlighting}{Verbatim}{commandchars=\\\{\}}
    % Add ',fontsize=\small' for more characters per line
    \newenvironment{Shaded}{}{}
    \newcommand{\KeywordTok}[1]{\textcolor[rgb]{0.00,0.44,0.13}{\textbf{{#1}}}}
    \newcommand{\DataTypeTok}[1]{\textcolor[rgb]{0.56,0.13,0.00}{{#1}}}
    \newcommand{\DecValTok}[1]{\textcolor[rgb]{0.25,0.63,0.44}{{#1}}}
    \newcommand{\BaseNTok}[1]{\textcolor[rgb]{0.25,0.63,0.44}{{#1}}}
    \newcommand{\FloatTok}[1]{\textcolor[rgb]{0.25,0.63,0.44}{{#1}}}
    \newcommand{\CharTok}[1]{\textcolor[rgb]{0.25,0.44,0.63}{{#1}}}
    \newcommand{\StringTok}[1]{\textcolor[rgb]{0.25,0.44,0.63}{{#1}}}
    \newcommand{\CommentTok}[1]{\textcolor[rgb]{0.38,0.63,0.69}{\textit{{#1}}}}
    \newcommand{\OtherTok}[1]{\textcolor[rgb]{0.00,0.44,0.13}{{#1}}}
    \newcommand{\AlertTok}[1]{\textcolor[rgb]{1.00,0.00,0.00}{\textbf{{#1}}}}
    \newcommand{\FunctionTok}[1]{\textcolor[rgb]{0.02,0.16,0.49}{{#1}}}
    \newcommand{\RegionMarkerTok}[1]{{#1}}
    \newcommand{\ErrorTok}[1]{\textcolor[rgb]{1.00,0.00,0.00}{\textbf{{#1}}}}
    \newcommand{\NormalTok}[1]{{#1}}
    
    % Additional commands for more recent versions of Pandoc
    \newcommand{\ConstantTok}[1]{\textcolor[rgb]{0.53,0.00,0.00}{{#1}}}
    \newcommand{\SpecialCharTok}[1]{\textcolor[rgb]{0.25,0.44,0.63}{{#1}}}
    \newcommand{\VerbatimStringTok}[1]{\textcolor[rgb]{0.25,0.44,0.63}{{#1}}}
    \newcommand{\SpecialStringTok}[1]{\textcolor[rgb]{0.73,0.40,0.53}{{#1}}}
    \newcommand{\ImportTok}[1]{{#1}}
    \newcommand{\DocumentationTok}[1]{\textcolor[rgb]{0.73,0.13,0.13}{\textit{{#1}}}}
    \newcommand{\AnnotationTok}[1]{\textcolor[rgb]{0.38,0.63,0.69}{\textbf{\textit{{#1}}}}}
    \newcommand{\CommentVarTok}[1]{\textcolor[rgb]{0.38,0.63,0.69}{\textbf{\textit{{#1}}}}}
    \newcommand{\VariableTok}[1]{\textcolor[rgb]{0.10,0.09,0.49}{{#1}}}
    \newcommand{\ControlFlowTok}[1]{\textcolor[rgb]{0.00,0.44,0.13}{\textbf{{#1}}}}
    \newcommand{\OperatorTok}[1]{\textcolor[rgb]{0.40,0.40,0.40}{{#1}}}
    \newcommand{\BuiltInTok}[1]{{#1}}
    \newcommand{\ExtensionTok}[1]{{#1}}
    \newcommand{\PreprocessorTok}[1]{\textcolor[rgb]{0.74,0.48,0.00}{{#1}}}
    \newcommand{\AttributeTok}[1]{\textcolor[rgb]{0.49,0.56,0.16}{{#1}}}
    \newcommand{\InformationTok}[1]{\textcolor[rgb]{0.38,0.63,0.69}{\textbf{\textit{{#1}}}}}
    \newcommand{\WarningTok}[1]{\textcolor[rgb]{0.38,0.63,0.69}{\textbf{\textit{{#1}}}}}
    
    
    % Define a nice break command that doesn't care if a line doesn't already
    % exist.
    \def\br{\hspace*{\fill} \\* }
    % Math Jax compatability definitions
    \def\gt{>}
    \def\lt{<}
    % Document parameters
    \title{assignment02\_dbac496}
    
    
    

    % Pygments definitions
    
\makeatletter
\def\PY@reset{\let\PY@it=\relax \let\PY@bf=\relax%
    \let\PY@ul=\relax \let\PY@tc=\relax%
    \let\PY@bc=\relax \let\PY@ff=\relax}
\def\PY@tok#1{\csname PY@tok@#1\endcsname}
\def\PY@toks#1+{\ifx\relax#1\empty\else%
    \PY@tok{#1}\expandafter\PY@toks\fi}
\def\PY@do#1{\PY@bc{\PY@tc{\PY@ul{%
    \PY@it{\PY@bf{\PY@ff{#1}}}}}}}
\def\PY#1#2{\PY@reset\PY@toks#1+\relax+\PY@do{#2}}

\expandafter\def\csname PY@tok@w\endcsname{\def\PY@tc##1{\textcolor[rgb]{0.73,0.73,0.73}{##1}}}
\expandafter\def\csname PY@tok@c\endcsname{\let\PY@it=\textit\def\PY@tc##1{\textcolor[rgb]{0.25,0.50,0.50}{##1}}}
\expandafter\def\csname PY@tok@cp\endcsname{\def\PY@tc##1{\textcolor[rgb]{0.74,0.48,0.00}{##1}}}
\expandafter\def\csname PY@tok@k\endcsname{\let\PY@bf=\textbf\def\PY@tc##1{\textcolor[rgb]{0.00,0.50,0.00}{##1}}}
\expandafter\def\csname PY@tok@kp\endcsname{\def\PY@tc##1{\textcolor[rgb]{0.00,0.50,0.00}{##1}}}
\expandafter\def\csname PY@tok@kt\endcsname{\def\PY@tc##1{\textcolor[rgb]{0.69,0.00,0.25}{##1}}}
\expandafter\def\csname PY@tok@o\endcsname{\def\PY@tc##1{\textcolor[rgb]{0.40,0.40,0.40}{##1}}}
\expandafter\def\csname PY@tok@ow\endcsname{\let\PY@bf=\textbf\def\PY@tc##1{\textcolor[rgb]{0.67,0.13,1.00}{##1}}}
\expandafter\def\csname PY@tok@nb\endcsname{\def\PY@tc##1{\textcolor[rgb]{0.00,0.50,0.00}{##1}}}
\expandafter\def\csname PY@tok@nf\endcsname{\def\PY@tc##1{\textcolor[rgb]{0.00,0.00,1.00}{##1}}}
\expandafter\def\csname PY@tok@nc\endcsname{\let\PY@bf=\textbf\def\PY@tc##1{\textcolor[rgb]{0.00,0.00,1.00}{##1}}}
\expandafter\def\csname PY@tok@nn\endcsname{\let\PY@bf=\textbf\def\PY@tc##1{\textcolor[rgb]{0.00,0.00,1.00}{##1}}}
\expandafter\def\csname PY@tok@ne\endcsname{\let\PY@bf=\textbf\def\PY@tc##1{\textcolor[rgb]{0.82,0.25,0.23}{##1}}}
\expandafter\def\csname PY@tok@nv\endcsname{\def\PY@tc##1{\textcolor[rgb]{0.10,0.09,0.49}{##1}}}
\expandafter\def\csname PY@tok@no\endcsname{\def\PY@tc##1{\textcolor[rgb]{0.53,0.00,0.00}{##1}}}
\expandafter\def\csname PY@tok@nl\endcsname{\def\PY@tc##1{\textcolor[rgb]{0.63,0.63,0.00}{##1}}}
\expandafter\def\csname PY@tok@ni\endcsname{\let\PY@bf=\textbf\def\PY@tc##1{\textcolor[rgb]{0.60,0.60,0.60}{##1}}}
\expandafter\def\csname PY@tok@na\endcsname{\def\PY@tc##1{\textcolor[rgb]{0.49,0.56,0.16}{##1}}}
\expandafter\def\csname PY@tok@nt\endcsname{\let\PY@bf=\textbf\def\PY@tc##1{\textcolor[rgb]{0.00,0.50,0.00}{##1}}}
\expandafter\def\csname PY@tok@nd\endcsname{\def\PY@tc##1{\textcolor[rgb]{0.67,0.13,1.00}{##1}}}
\expandafter\def\csname PY@tok@s\endcsname{\def\PY@tc##1{\textcolor[rgb]{0.73,0.13,0.13}{##1}}}
\expandafter\def\csname PY@tok@sd\endcsname{\let\PY@it=\textit\def\PY@tc##1{\textcolor[rgb]{0.73,0.13,0.13}{##1}}}
\expandafter\def\csname PY@tok@si\endcsname{\let\PY@bf=\textbf\def\PY@tc##1{\textcolor[rgb]{0.73,0.40,0.53}{##1}}}
\expandafter\def\csname PY@tok@se\endcsname{\let\PY@bf=\textbf\def\PY@tc##1{\textcolor[rgb]{0.73,0.40,0.13}{##1}}}
\expandafter\def\csname PY@tok@sr\endcsname{\def\PY@tc##1{\textcolor[rgb]{0.73,0.40,0.53}{##1}}}
\expandafter\def\csname PY@tok@ss\endcsname{\def\PY@tc##1{\textcolor[rgb]{0.10,0.09,0.49}{##1}}}
\expandafter\def\csname PY@tok@sx\endcsname{\def\PY@tc##1{\textcolor[rgb]{0.00,0.50,0.00}{##1}}}
\expandafter\def\csname PY@tok@m\endcsname{\def\PY@tc##1{\textcolor[rgb]{0.40,0.40,0.40}{##1}}}
\expandafter\def\csname PY@tok@gh\endcsname{\let\PY@bf=\textbf\def\PY@tc##1{\textcolor[rgb]{0.00,0.00,0.50}{##1}}}
\expandafter\def\csname PY@tok@gu\endcsname{\let\PY@bf=\textbf\def\PY@tc##1{\textcolor[rgb]{0.50,0.00,0.50}{##1}}}
\expandafter\def\csname PY@tok@gd\endcsname{\def\PY@tc##1{\textcolor[rgb]{0.63,0.00,0.00}{##1}}}
\expandafter\def\csname PY@tok@gi\endcsname{\def\PY@tc##1{\textcolor[rgb]{0.00,0.63,0.00}{##1}}}
\expandafter\def\csname PY@tok@gr\endcsname{\def\PY@tc##1{\textcolor[rgb]{1.00,0.00,0.00}{##1}}}
\expandafter\def\csname PY@tok@ge\endcsname{\let\PY@it=\textit}
\expandafter\def\csname PY@tok@gs\endcsname{\let\PY@bf=\textbf}
\expandafter\def\csname PY@tok@gp\endcsname{\let\PY@bf=\textbf\def\PY@tc##1{\textcolor[rgb]{0.00,0.00,0.50}{##1}}}
\expandafter\def\csname PY@tok@go\endcsname{\def\PY@tc##1{\textcolor[rgb]{0.53,0.53,0.53}{##1}}}
\expandafter\def\csname PY@tok@gt\endcsname{\def\PY@tc##1{\textcolor[rgb]{0.00,0.27,0.87}{##1}}}
\expandafter\def\csname PY@tok@err\endcsname{\def\PY@bc##1{\setlength{\fboxsep}{0pt}\fcolorbox[rgb]{1.00,0.00,0.00}{1,1,1}{\strut ##1}}}
\expandafter\def\csname PY@tok@kc\endcsname{\let\PY@bf=\textbf\def\PY@tc##1{\textcolor[rgb]{0.00,0.50,0.00}{##1}}}
\expandafter\def\csname PY@tok@kd\endcsname{\let\PY@bf=\textbf\def\PY@tc##1{\textcolor[rgb]{0.00,0.50,0.00}{##1}}}
\expandafter\def\csname PY@tok@kn\endcsname{\let\PY@bf=\textbf\def\PY@tc##1{\textcolor[rgb]{0.00,0.50,0.00}{##1}}}
\expandafter\def\csname PY@tok@kr\endcsname{\let\PY@bf=\textbf\def\PY@tc##1{\textcolor[rgb]{0.00,0.50,0.00}{##1}}}
\expandafter\def\csname PY@tok@bp\endcsname{\def\PY@tc##1{\textcolor[rgb]{0.00,0.50,0.00}{##1}}}
\expandafter\def\csname PY@tok@fm\endcsname{\def\PY@tc##1{\textcolor[rgb]{0.00,0.00,1.00}{##1}}}
\expandafter\def\csname PY@tok@vc\endcsname{\def\PY@tc##1{\textcolor[rgb]{0.10,0.09,0.49}{##1}}}
\expandafter\def\csname PY@tok@vg\endcsname{\def\PY@tc##1{\textcolor[rgb]{0.10,0.09,0.49}{##1}}}
\expandafter\def\csname PY@tok@vi\endcsname{\def\PY@tc##1{\textcolor[rgb]{0.10,0.09,0.49}{##1}}}
\expandafter\def\csname PY@tok@vm\endcsname{\def\PY@tc##1{\textcolor[rgb]{0.10,0.09,0.49}{##1}}}
\expandafter\def\csname PY@tok@sa\endcsname{\def\PY@tc##1{\textcolor[rgb]{0.73,0.13,0.13}{##1}}}
\expandafter\def\csname PY@tok@sb\endcsname{\def\PY@tc##1{\textcolor[rgb]{0.73,0.13,0.13}{##1}}}
\expandafter\def\csname PY@tok@sc\endcsname{\def\PY@tc##1{\textcolor[rgb]{0.73,0.13,0.13}{##1}}}
\expandafter\def\csname PY@tok@dl\endcsname{\def\PY@tc##1{\textcolor[rgb]{0.73,0.13,0.13}{##1}}}
\expandafter\def\csname PY@tok@s2\endcsname{\def\PY@tc##1{\textcolor[rgb]{0.73,0.13,0.13}{##1}}}
\expandafter\def\csname PY@tok@sh\endcsname{\def\PY@tc##1{\textcolor[rgb]{0.73,0.13,0.13}{##1}}}
\expandafter\def\csname PY@tok@s1\endcsname{\def\PY@tc##1{\textcolor[rgb]{0.73,0.13,0.13}{##1}}}
\expandafter\def\csname PY@tok@mb\endcsname{\def\PY@tc##1{\textcolor[rgb]{0.40,0.40,0.40}{##1}}}
\expandafter\def\csname PY@tok@mf\endcsname{\def\PY@tc##1{\textcolor[rgb]{0.40,0.40,0.40}{##1}}}
\expandafter\def\csname PY@tok@mh\endcsname{\def\PY@tc##1{\textcolor[rgb]{0.40,0.40,0.40}{##1}}}
\expandafter\def\csname PY@tok@mi\endcsname{\def\PY@tc##1{\textcolor[rgb]{0.40,0.40,0.40}{##1}}}
\expandafter\def\csname PY@tok@il\endcsname{\def\PY@tc##1{\textcolor[rgb]{0.40,0.40,0.40}{##1}}}
\expandafter\def\csname PY@tok@mo\endcsname{\def\PY@tc##1{\textcolor[rgb]{0.40,0.40,0.40}{##1}}}
\expandafter\def\csname PY@tok@ch\endcsname{\let\PY@it=\textit\def\PY@tc##1{\textcolor[rgb]{0.25,0.50,0.50}{##1}}}
\expandafter\def\csname PY@tok@cm\endcsname{\let\PY@it=\textit\def\PY@tc##1{\textcolor[rgb]{0.25,0.50,0.50}{##1}}}
\expandafter\def\csname PY@tok@cpf\endcsname{\let\PY@it=\textit\def\PY@tc##1{\textcolor[rgb]{0.25,0.50,0.50}{##1}}}
\expandafter\def\csname PY@tok@c1\endcsname{\let\PY@it=\textit\def\PY@tc##1{\textcolor[rgb]{0.25,0.50,0.50}{##1}}}
\expandafter\def\csname PY@tok@cs\endcsname{\let\PY@it=\textit\def\PY@tc##1{\textcolor[rgb]{0.25,0.50,0.50}{##1}}}

\def\PYZbs{\char`\\}
\def\PYZus{\char`\_}
\def\PYZob{\char`\{}
\def\PYZcb{\char`\}}
\def\PYZca{\char`\^}
\def\PYZam{\char`\&}
\def\PYZlt{\char`\<}
\def\PYZgt{\char`\>}
\def\PYZsh{\char`\#}
\def\PYZpc{\char`\%}
\def\PYZdl{\char`\$}
\def\PYZhy{\char`\-}
\def\PYZsq{\char`\'}
\def\PYZdq{\char`\"}
\def\PYZti{\char`\~}
% for compatibility with earlier versions
\def\PYZat{@}
\def\PYZlb{[}
\def\PYZrb{]}
\makeatother


    % Exact colors from NB
    \definecolor{incolor}{rgb}{0.0, 0.0, 0.5}
    \definecolor{outcolor}{rgb}{0.545, 0.0, 0.0}



    
    % Prevent overflowing lines due to hard-to-break entities
    \sloppy 
    % Setup hyperref package
    \hypersetup{
      breaklinks=true,  % so long urls are correctly broken across lines
      colorlinks=true,
      urlcolor=urlcolor,
      linkcolor=linkcolor,
      citecolor=citecolor,
      }
    % Slightly bigger margins than the latex defaults
    
    \geometry{verbose,tmargin=1in,bmargin=1in,lmargin=1in,rmargin=1in}
    
    

    \begin{document}
    
    
    \maketitle
    
    

    
    \textbf{Due:} 2018.09.24 \textbf{\textbar{} Author:} Dinh Che
\textbf{\textbar{} Student ID:} 5721970 \textbf{\textbar{} Email:}
dbac496@uowmail.edu.au

    \hypertarget{question-1.}{%
\subsubsection{Question 1.}\label{question-1.}}

\begin{itemize}
\tightlist
\item
  Expected number of hashes:

  \begin{itemize}
  \tightlist
  \item
    \(h =\sim 2^{k-1}\)
  \end{itemize}
\item
  Cost of completing \(P\):

  \begin{itemize}
  \tightlist
  \item
    \(\sim m \cdot 2^{k-1}\)
  \end{itemize}
\end{itemize}

    \begin{Verbatim}[commandchars=\\\{\}]
{\color{incolor}In [{\color{incolor} }]:} \PY{o}{\PYZpc{}} \PY{n}{matplotlib} \PY{n}{inline}
        \PY{k+kn}{import} \PY{n+nn}{numpy} \PY{k}{as} \PY{n+nn}{np}
        \PY{k+kn}{import} \PY{n+nn}{matplotlib}\PY{n+nn}{.}\PY{n+nn}{pyplot} \PY{k}{as} \PY{n+nn}{plt}
        \PY{k+kn}{import} \PY{n+nn}{pandas} \PY{k}{as} \PY{n+nn}{pd}
        \PY{k+kn}{import} \PY{n+nn}{scipy}\PY{n+nn}{.}\PY{n+nn}{stats} \PY{k}{as} \PY{n+nn}{stats}
        
        
        \PY{k}{def} \PY{n+nf}{cost}\PY{p}{(}\PY{n}{m}\PY{p}{,} \PY{n}{k}\PY{p}{)}\PY{p}{:}
            \PY{n}{cases} \PY{o}{=} \PY{n}{m} \PY{o}{*} \PY{n}{np}\PY{o}{.}\PY{n}{power}\PY{p}{(}\PY{l+m+mi}{2}\PY{p}{,} \PY{n}{k} \PY{o}{\PYZhy{}} \PY{l+m+mi}{1}\PY{p}{)}
            \PY{k}{return} \PY{n}{cases}
        
        
        \PY{n}{case\PYZus{}2} \PY{o}{=} \PY{p}{[}
            \PY{p}{(}\PY{l+s+s1}{\PYZsq{}}\PY{l+s+s1}{\PYZsh{} Hashes}\PY{l+s+s1}{\PYZsq{}}\PY{p}{,} \PY{p}{[}\PY{n}{x} \PY{k}{for} \PY{n}{x} \PY{o+ow}{in} \PY{n+nb}{range}\PY{p}{(}\PY{l+m+mi}{1}\PY{p}{,} \PY{l+m+mi}{17}\PY{p}{)}\PY{p}{]}\PY{p}{)}\PY{p}{,}
            \PY{p}{(}\PY{l+s+s1}{\PYZsq{}}\PY{l+s+s1}{Case 2}\PY{l+s+s1}{\PYZsq{}}\PY{p}{,} \PY{p}{[}\PY{l+m+mi}{0}\PY{p}{,} \PY{l+m+mi}{0}\PY{p}{,} \PY{l+m+mi}{0}\PY{p}{,} \PY{l+m+mi}{1}\PY{p}{,} \PY{l+m+mi}{4}\PY{p}{,} \PY{l+m+mi}{10}\PY{p}{,} \PY{l+m+mi}{20}\PY{p}{,} \PY{l+m+mi}{31}\PY{p}{,} \PY{l+m+mi}{40}\PY{p}{,} \PY{l+m+mi}{44}\PY{p}{,} \PY{l+m+mi}{40}\PY{p}{,} \PY{l+m+mi}{31}\PY{p}{,} \PY{l+m+mi}{20}\PY{p}{,} \PY{l+m+mi}{10}\PY{p}{,} \PY{l+m+mi}{4}\PY{p}{,} \PY{l+m+mi}{1}\PY{p}{]}\PY{p}{)}
        \PY{p}{]}
        
        \PY{n}{df\PYZus{}2} \PY{o}{=} \PY{n}{pd}\PY{o}{.}\PY{n}{DataFrame}\PY{o}{.}\PY{n}{from\PYZus{}dict}\PY{p}{(}\PY{n+nb}{dict}\PY{p}{(}\PY{n}{case\PYZus{}2}\PY{p}{)}\PY{p}{)}
        
        \PY{n}{data} \PY{o}{=} \PY{p}{[}
            \PY{p}{(}\PY{l+s+s1}{\PYZsq{}}\PY{l+s+s1}{Case 1}\PY{l+s+s1}{\PYZsq{}}\PY{p}{,} \PY{p}{[}\PY{l+m+mi}{1}\PY{p}{]} \PY{o}{*} \PY{l+m+mi}{16}\PY{p}{)}\PY{p}{,}
            \PY{p}{(}\PY{l+s+s1}{\PYZsq{}}\PY{l+s+s1}{Case 2}\PY{l+s+s1}{\PYZsq{}}\PY{p}{,} \PY{p}{[}\PY{l+m+mi}{0}\PY{p}{,} \PY{l+m+mi}{0}\PY{p}{,} \PY{l+m+mi}{0}\PY{p}{,} \PY{l+m+mi}{1}\PY{p}{,} \PY{l+m+mi}{4}\PY{p}{,} \PY{l+m+mi}{10}\PY{p}{,} \PY{l+m+mi}{20}\PY{p}{,} \PY{l+m+mi}{31}\PY{p}{,} \PY{l+m+mi}{40}\PY{p}{,} \PY{l+m+mi}{44}\PY{p}{,} \PY{l+m+mi}{40}\PY{p}{,} \PY{l+m+mi}{31}\PY{p}{,} \PY{l+m+mi}{20}\PY{p}{,} \PY{l+m+mi}{10}\PY{p}{,} \PY{l+m+mi}{4}\PY{p}{,} \PY{l+m+mi}{1}\PY{p}{]}\PY{p}{)}\PY{p}{,}
            \PY{p}{(}\PY{l+s+s1}{\PYZsq{}}\PY{l+s+s1}{Case 3}\PY{l+s+s1}{\PYZsq{}}\PY{p}{,} \PY{p}{[}\PY{n}{cost}\PY{p}{(}\PY{n}{x}\PY{p}{,} \PY{l+m+mi}{6}\PY{p}{)} \PY{k}{for} \PY{n}{x} \PY{o+ow}{in} \PY{n+nb}{range}\PY{p}{(}\PY{l+m+mi}{1}\PY{p}{,} \PY{l+m+mi}{17}\PY{p}{)}\PY{p}{]}\PY{p}{)}
        \PY{p}{]}
        
        \PY{n}{df\PYZus{}1} \PY{o}{=} \PY{n}{pd}\PY{o}{.}\PY{n}{DataFrame}\PY{o}{.}\PY{n}{from\PYZus{}dict}\PY{p}{(}\PY{n+nb}{dict}\PY{p}{(}\PY{n}{data}\PY{p}{)}\PY{p}{)}
        
        \PY{n+nb}{print}\PY{p}{(}\PY{n}{df\PYZus{}2}\PY{o}{.}\PY{n}{T}\PY{p}{)}
        
        \PY{n}{plt}\PY{o}{.}\PY{n}{figure}\PY{p}{(}\PY{p}{)}
        \PY{n}{df\PYZus{}2}\PY{o}{.}\PY{n}{plot}\PY{p}{(}\PY{n}{kind}\PY{o}{=}\PY{l+s+s1}{\PYZsq{}}\PY{l+s+s1}{scatter}\PY{l+s+s1}{\PYZsq{}}\PY{p}{,} \PY{n}{title}\PY{o}{=}\PY{l+s+s1}{\PYZsq{}}\PY{l+s+s1}{Hashes vs. Cases}\PY{l+s+s1}{\PYZsq{}}\PY{p}{,} \PY{n}{x}\PY{o}{=}\PY{l+s+s1}{\PYZsq{}}\PY{l+s+s1}{\PYZsh{} Hashes}\PY{l+s+s1}{\PYZsq{}}\PY{p}{,} \PY{n}{y}\PY{o}{=}\PY{l+s+s1}{\PYZsq{}}\PY{l+s+s1}{Case 2}\PY{l+s+s1}{\PYZsq{}}\PY{p}{)}
        
        \PY{n}{plt}\PY{o}{.}\PY{n}{figure}\PY{p}{(}\PY{p}{)}
        \PY{n}{df\PYZus{}2}\PY{o}{.}\PY{n}{plot}\PY{p}{(}\PY{n}{kind}\PY{o}{=}\PY{l+s+s1}{\PYZsq{}}\PY{l+s+s1}{line}\PY{l+s+s1}{\PYZsq{}}\PY{p}{,} \PY{n}{title}\PY{o}{=}\PY{l+s+s1}{\PYZsq{}}\PY{l+s+s1}{Hashes vs. Cases}\PY{l+s+s1}{\PYZsq{}}\PY{p}{,} \PY{n}{x}\PY{o}{=}\PY{l+s+s1}{\PYZsq{}}\PY{l+s+s1}{\PYZsh{} Hashes}\PY{l+s+s1}{\PYZsq{}}\PY{p}{,} \PY{n}{y}\PY{o}{=}\PY{l+s+s1}{\PYZsq{}}\PY{l+s+s1}{Case 2}\PY{l+s+s1}{\PYZsq{}}\PY{p}{)}
        
        \PY{n}{plt}\PY{o}{.}\PY{n}{figure}\PY{p}{(}\PY{p}{)}
        \PY{n}{df\PYZus{}2}\PY{p}{[}\PY{l+s+s1}{\PYZsq{}}\PY{l+s+s1}{Case 2}\PY{l+s+s1}{\PYZsq{}}\PY{p}{]}\PY{o}{.}\PY{n}{plot}\PY{p}{(}\PY{n}{kind}\PY{o}{=}\PY{l+s+s1}{\PYZsq{}}\PY{l+s+s1}{box}\PY{l+s+s1}{\PYZsq{}}\PY{p}{,} \PY{n}{title}\PY{o}{=}\PY{l+s+s1}{\PYZsq{}}\PY{l+s+s1}{Case 2 Distribution}\PY{l+s+s1}{\PYZsq{}}\PY{p}{)}
        \PY{n}{plt}\PY{o}{.}\PY{n}{figure}\PY{p}{(}\PY{p}{)}
        \PY{n}{df\PYZus{}2}\PY{p}{[}\PY{l+s+s1}{\PYZsq{}}\PY{l+s+s1}{Case 2}\PY{l+s+s1}{\PYZsq{}}\PY{p}{]}\PY{o}{.}\PY{n}{hist}\PY{p}{(}\PY{n}{bins}\PY{o}{=}\PY{l+m+mi}{16}\PY{p}{)}
        
        \PY{n}{plt}\PY{o}{.}\PY{n}{show}\PY{p}{(}\PY{p}{)}
        \PY{n+nb}{print}\PY{p}{(}\PY{n}{df\PYZus{}2}\PY{p}{[}\PY{l+s+s1}{\PYZsq{}}\PY{l+s+s1}{Case 2}\PY{l+s+s1}{\PYZsq{}}\PY{p}{]}\PY{o}{.}\PY{n}{describe}\PY{p}{(}\PY{p}{)}\PY{p}{)}
        \PY{n+nb}{print}\PY{p}{(}\PY{l+s+s1}{\PYZsq{}}\PY{l+s+s1}{median: }\PY{l+s+s1}{\PYZsq{}} \PY{o}{+} \PY{n+nb}{str}\PY{p}{(}\PY{n}{df\PYZus{}2}\PY{p}{[}\PY{l+s+s1}{\PYZsq{}}\PY{l+s+s1}{Case 2}\PY{l+s+s1}{\PYZsq{}}\PY{p}{]}\PY{o}{.}\PY{n}{median}\PY{p}{(}\PY{p}{)}\PY{p}{)}\PY{p}{)}
\end{Verbatim}


    \hypertarget{question-2.}{%
\subsubsection{Question 2.}\label{question-2.}}

\begin{itemize}
\tightlist
\item
  Original
\end{itemize}

\begin{verbatim}
permit = CheckAccess()

IF (permit == Access_Denied)
    Print "Access Denied"
ELSE 
    Print "Access Granted"
    Run Function()
\end{verbatim}

\begin{itemize}
\tightlist
\item
  "Default deny, not default allow!
\end{itemize}

\begin{verbatim}
permit = CheckAccess()
IF (permit == Access_Granted)
    Print "Access Granted"
    Run Function()
ELSE
    Print "Access Denied"
\end{verbatim}

    \hypertarget{question-3.}{%
\subsubsection{Question 3.}\label{question-3.}}

Consider that the incidence of viral attachments in email messages is 1
in 800. Your malware checker will correctly identify a message as viral
95\% of the time. Your malware checker will correctly identify a message
non-viral 95\% of the time. Your malware checker has just flagged a
message as being malware. What is the probability that the message is
actually okay? Justify your answer using Bayes theorem

\[
\begin{align*}
& P
\end{align*}
\]

    \hypertarget{question-4.}{%
\subsubsection{Question 4.}\label{question-4.}}

Describe, in your own words, a specific instance of an insider placing
malware within a system. You should describe the type of malware placed,
the expected likely impact, and some details regarding the outcome. This
is not meaning a hypothetical scenario you have made up, find an actual
real world example.

    \hypertarget{references}{%
\paragraph{References:}\label{references}}

\href{https://www.ted.com/talks/ralph_langner_cracking_stuxnet_a_21st_century_cyberweapon}{x}
K. Zetter, ``How Digital Detectives Deciphered Stuxnet, The Most
Menacing Malware in History,'' \emph{wired.com,} Jul.~07, 2011.
{[}Online{]} Available:
https://www.wired.com/2011/07/how-digital-detectives-deciphered-stuxnet/
{[}Accessed: Sep.~21, 2018{]}.

\begin{itemize}
\item
  ``Months earlier, in June 2009, someone had silently unleashed a
  sophisticated and destructive digital worm that had been slithering
  its way through computers in Iran with just one aim - to sabotage the
  country's uranium enrichment program and prevent President Mahmoud
  Ahmadinejad from building a nuclear weapon.''
\item
  ``Ulasen's research team got hold of the virus infecting their
  client's computer and realized it was using a''zero -day" exploit to
  spread. Zero-days are the hacking world's most potent weapons: They
  exploit vulnerabilities in software that are yet unknown to the
  software maker or antivirus vendors."
\item
\end{itemize}

\href{https://www.ted.com/talks/ralph_langner_cracking_stuxnet_a_21st_century_cyberweapon}{x}
Wikimedia Foundation Inc, ``Stuxnet,'' \emph{Wikipedia}, 2018.
{[}Online{]}. Available: https://en.wikipedia.org/wiki/Stuxnet
{[}Accessed: Sep.~21, 2018{]}.

\begin{itemize}
\item
\end{itemize}

\href{https://www.ted.com/talks/ralph_langner_cracking_stuxnet_a_21st_century_cyberweapon}{x}
J. Shearer, ``W32.Stuxnet Writeup,'' \emph{Symantec Enterprise,}
Security Center, Sep.~26, 2017. {[}Online{]}. Available:
https://www.symantec.com/security-center/writeup/2010-071400-3123-99
{[}Accessed: Sep.~21, 2018{]}.

\begin{itemize}
\item
\end{itemize}

\href{https://www.ted.com/talks/ralph_langner_cracking_stuxnet_a_21st_century_cyberweapon}{x}
D. Kushner, ``The Real Story of Stuxnet,'' \emph{IEEE Spectrum,}
Feb.~26, 2013. {[}Online{]}. Available:
https://spectrum.ieee.org/telecom/security/the-real-story-of-stuxnet
{[}Accessed: Sep.~21, 2018{]}.

\begin{itemize}
\item
\end{itemize}

\href{https://www.ted.com/talks/ralph_langner_cracking_stuxnet_a_21st_century_cyberweapon}{x}
R. Naraine, ``Stuxnet attackers used 4 Windows zero-day exploits,''
\emph{zdnet.com}, Sep.~24, 2010. {[}Online{]}. Available:
https://www.zdnet.com/article/stuxnet-attackers-used-4-windows-zero-day-exploits/
{[}Accessed: Sep.~21, 2018{]}.

\begin{itemize}
\tightlist
\item
  ``\ldots{} four different zero-day security vulnerabilities to burrow
  into -- and spread around -- Microsoft's Windows operating system.''
\item
  ``\ldots{} initially targeted the old MS08-067 vulnerability (used in
  the Conficker attack), a new LNK (Windows Shortcut) flaw to launch
  exploit code on vulnerable WIndows systems and a zero-day bug in the
  Print Spooler Service that makes it possible for malicious code to be
  passed to, and then executed on, a remote machine.''
\item
  ``\ldots{} worm also used signed digital certificates stolen from
  RealTek and JMicron and also exploited security problems in the
  Simatic WinCC SCADA systems.''
\end{itemize}

\href{https://www.ted.com/talks/ralph_langner_cracking_stuxnet_a_21st_century_cyberweapon}{x}
J. Fruhlinger, ``What is Stuxnet, who created it and how does it work?''
\emph{csoonline.com}, Aug.~22, 2017. {[}Online{]}. Available:
https://www.csoonline.com/article/3218104/malware/what-is-stuxnet-who-created-it-and-how-does-it-work.html
{[}Accessed: Sep.~21, 2018{]}

\begin{itemize}
\tightlist
\item
  ``Stux net was first identified by the infosec community in 2010, but
  development on it probably began in 2005.''
\item
  ``Despite its unparalleled ability to spread and its widespread
  infection rate, Stuxnet does little or no harm to computers not
  involved in uranium enrichment. When it infects a computer, it checks
  to see if that computer is connected to specific models of
  programmable logic controllers (PLCs) manufactured by Siemens. PLCs
  are how computers interact with and control industrial machinery like
  uranium centrifuges.''
\item
  ``The worm then alters PLCs' programming, resulting in the centrifuges
  being spun too quickly and for too long, damaging or destroying the
  delicate equipment in the process.''
\item
  ``While this is happening, the PLCs tell the controller computer that
  everything is working fine, making it difficult to detect or diagnose
  what's going wrong until it's too late.''
\end{itemize}

\href{https://www.ted.com/talks/ralph_langner_cracking_stuxnet_a_21st_century_cyberweapon}{x}
D. Sanger, ``Obama Order Sped Up Wave of Cyberattacks Against Iran,''
\emph{nytimes.com}, Jun.~1, 2012. {[}Online{]}. Available:
https://www.nytimes.com/2012/06/01/world/middleeast/obama-ordered-wave-of-cyberattacks-against-iran.html?pagewanted=1\&\_r=1\&hp
{[}Accessed: Sep.~22, 2018{]}.

\begin{itemize}
\item
\end{itemize}

    \hypertarget{question-5.}{%
\subsubsection{Question 5.}\label{question-5.}}

In the context of phishing, list 8 points that can be used in checking
the legitimacy of an email. Justify why each is appropriate as an
indicator. Note that some points could relate to characteristics of
legitimate messages, and others could be indicators of a phishing
message.

    In relation to detecting phishing attempts, the following are points to
consider when checking the legitimacy of an email:

\begin{enumerate}
\def\labelenumi{\arabic{enumi}.}
\item
  Anti-spam filtering software

  The use of anti-spam filtering software at the desktop level or at the
  gateway level (by ISPs and email service providers) is considered
\item
  Identifying receiving email address in email header
\item
  Check any domain and URL link sent in email body for spelling errors

  \begin{itemize}
  \tightlist
  \item
    Additionally, check the web address of any embedded links
  \end{itemize}
\item
  Type desired web address into browser instead of clicking on embedded
  links
\item
  Confirm the apparent sending institution is a legitimate provider of
  service
\item
  Check for any spelling or grammatical errors in the email body
  content\\
\item
  Call the apparent sending institution to verify authenticity\\
\item
  Do not open any attachments that you are not expecting and especially
  if it is an executable (.exe suffix)
\item
  Check the whether the website is using the HTTPS protocol and check
  the SSL certificate is legitimate
\end{enumerate}

{[}x{]} G. Tally, R. Thomas, \& T. Van Vleck, ``Anti-Phishing: Best
Practices for Institutions and Consumers,'' \emph{McAfee}, Sep.~2004.
{[}Online{]}. Available:
https://docs.apwg.org/sponsors\_technical\_papers/Anti-Phishing\_Best\_Practices\_for\_Institutions\_Consumer0904.pdf
{[}Accessed: Sep, 23, 2018{]}.

{[}x{]} W. Zamora, ``Something's phishy: How to detect phishing
attempts,'' \emph{Malwarebytes Labs}, Jun.~26, 2017. {[}Online{]}.
Available:
https://blog.malwarebytes.com/101/2017/06/somethings-phishy-how-to-detect-phishing-attempts/
{[}Accessed: Sep.~23, 2018{]}.

{[}x{]} Anti-Phishing Working Group, ``How to Avoid Phishing Scams,''
\emph{Anti-Phishing Working Group}. {[}Online{]}. Available:
https://apwg.org/resources/overview/avoid-phishing-scams {[}Accessed:
Sep.~23, 2018{]}

    \hypertarget{question-6.}{%
\subsubsection{Question 6.}\label{question-6.}}

    \begin{Verbatim}[commandchars=\\\{\}]
{\color{incolor}In [{\color{incolor}28}]:} \PY{k+kn}{import} \PY{n+nn}{random}
         \PY{k+kn}{from} \PY{n+nn}{enum} \PY{k}{import} \PY{n}{Enum}
         \PY{k+kn}{from} \PY{n+nn}{IPython}\PY{n+nn}{.}\PY{n+nn}{display} \PY{k}{import} \PY{n}{display}\PY{p}{,} \PY{n}{HTML}
         \PY{k+kn}{import} \PY{n+nn}{numpy} \PY{k}{as} \PY{n+nn}{np}
         \PY{k+kn}{import} \PY{n+nn}{pandas} \PY{k}{as} \PY{n+nn}{pd}
         
         
         \PY{k}{class} \PY{n+nc}{State}\PY{p}{(}\PY{n}{Enum}\PY{p}{)}\PY{p}{:}
             \PY{n}{UN\PYZus{}INF} \PY{o}{=} \PY{l+m+mi}{0}
             \PY{n}{X\PYZus{}INF} \PY{o}{=} \PY{l+m+mi}{1}
             \PY{n}{W\PYZus{}INF} \PY{o}{=} \PY{l+m+mi}{2}
         
         
         \PY{n}{time} \PY{o}{=} \PY{l+m+mi}{0}
         \PY{n}{n\PYZus{}C} \PY{o}{=} \PY{l+m+mi}{0}  \PY{c+c1}{\PYZsh{} number of computers}
         \PY{n}{n\PYZus{}X} \PY{o}{=} \PY{l+m+mi}{0}  \PY{c+c1}{\PYZsh{} number of X infected computers}
         
         \PY{c+c1}{\PYZsh{} Get the number of computers}
         \PY{n}{n\PYZus{}C} \PY{o}{=} \PY{l+m+mi}{10}
         
         \PY{n}{computers} \PY{o}{=} \PY{p}{\PYZob{}}\PY{n}{x}\PY{p}{:} \PY{n}{State}\PY{o}{.}\PY{n}{UN\PYZus{}INF} \PY{k}{for} \PY{n}{x} \PY{o+ow}{in} \PY{n+nb}{range}\PY{p}{(}\PY{l+m+mi}{1}\PY{p}{,} \PY{n}{n\PYZus{}C} \PY{o}{+} \PY{l+m+mi}{1}\PY{p}{)}\PY{p}{\PYZcb{}}
         
         \PY{n}{random}\PY{o}{.}\PY{n}{seed}\PY{p}{(}\PY{l+m+mi}{2}\PY{p}{)}
         \PY{n}{uninfected\PYZus{}sample} \PY{o}{=} \PY{n}{random}\PY{o}{.}\PY{n}{sample}\PY{p}{(}\PY{n+nb}{list}\PY{p}{(}\PY{n}{computers}\PY{o}{.}\PY{n}{keys}\PY{p}{(}\PY{p}{)}\PY{p}{)}\PY{p}{,} \PY{n}{n\PYZus{}C}\PY{p}{)}  \PY{c+c1}{\PYZsh{} set up a random sample with no duplicates}
         
         \PY{n}{choice} \PY{o}{=} \PY{n}{random}\PY{o}{.}\PY{n}{choice}\PY{p}{(}\PY{n}{uninfected\PYZus{}sample}\PY{p}{)}  \PY{c+c1}{\PYZsh{} select one to be infected at the beginning}
         \PY{n}{computers}\PY{p}{[}\PY{n}{uninfected\PYZus{}sample}\PY{o}{.}\PY{n}{pop}\PY{p}{(}\PY{p}{)}\PY{p}{]} \PY{o}{=} \PY{n}{State}\PY{o}{.}\PY{n}{X\PYZus{}INF}
         
         \PY{c+c1}{\PYZsh{} Set pandas display options to fit width}
         \PY{n}{pd}\PY{o}{.}\PY{n}{set\PYZus{}option}\PY{p}{(}\PY{l+s+s1}{\PYZsq{}}\PY{l+s+s1}{display.max\PYZus{}rows}\PY{l+s+s1}{\PYZsq{}}\PY{p}{,} \PY{l+m+mi}{100}\PY{p}{)}
         \PY{n}{pd}\PY{o}{.}\PY{n}{set\PYZus{}option}\PY{p}{(}\PY{l+s+s1}{\PYZsq{}}\PY{l+s+s1}{display.max\PYZus{}columns}\PY{l+s+s1}{\PYZsq{}}\PY{p}{,} \PY{l+m+mi}{100}\PY{p}{)}
         \PY{n}{pd}\PY{o}{.}\PY{n}{set\PYZus{}option}\PY{p}{(}\PY{l+s+s1}{\PYZsq{}}\PY{l+s+s1}{display.width}\PY{l+s+s1}{\PYZsq{}}\PY{p}{,} \PY{l+m+mi}{1000}\PY{p}{)}
         
         \PY{k}{for} \PY{n}{i} \PY{o+ow}{in} \PY{n}{np}\PY{o}{.}\PY{n}{arange}\PY{p}{(}\PY{l+m+mi}{0}\PY{p}{,} \PY{l+m+mf}{24.5}\PY{p}{,} \PY{l+m+mf}{0.5}\PY{p}{)}\PY{p}{:}
             \PY{n}{time} \PY{o}{=} \PY{n}{i}
         
             \PY{c+c1}{\PYZsh{} every hour the worm X spreads from each infected computer to one previously uninfected computer.}
             \PY{k}{if} \PY{n}{i} \PY{o}{\PYZpc{}} \PY{l+m+mi}{1} \PY{o}{==} \PY{l+m+mi}{0}\PY{p}{:}
                 \PY{n}{infected} \PY{o}{=} \PY{p}{[}\PY{n}{val} \PY{k}{for} \PY{n}{\PYZus{}}\PY{p}{,} \PY{n}{val} \PY{o+ow}{in} \PY{n}{computers}\PY{o}{.}\PY{n}{items}\PY{p}{(}\PY{p}{)} \PY{k}{if} \PY{n}{val} \PY{o}{==} \PY{n}{State}\PY{o}{.}\PY{n}{X\PYZus{}INF}\PY{p}{]}
                 \PY{n}{n\PYZus{}X} \PY{o}{=} \PY{n+nb}{len}\PY{p}{(}\PY{n}{infected}\PY{p}{)}
         
                 \PY{k}{if} \PY{n}{uninfected\PYZus{}sample} \PY{o+ow}{is} \PY{k+kc}{None}\PY{p}{:}
                     \PY{k}{continue}
                 \PY{k}{elif} \PY{n}{n\PYZus{}X} \PY{o}{\PYZlt{}} \PY{n+nb}{len}\PY{p}{(}\PY{n}{uninfected\PYZus{}sample}\PY{p}{)}\PY{p}{:}
                     \PY{k}{for} \PY{n}{j} \PY{o+ow}{in} \PY{n+nb}{range}\PY{p}{(}\PY{n}{n\PYZus{}X}\PY{p}{)}\PY{p}{:}
                         \PY{n}{computers}\PY{p}{[}\PY{n}{uninfected\PYZus{}sample}\PY{o}{.}\PY{n}{pop}\PY{p}{(}\PY{p}{)}\PY{p}{]} \PY{o}{=} \PY{n}{State}\PY{o}{.}\PY{n}{X\PYZus{}INF}
                 \PY{k}{else}\PY{p}{:}
                     \PY{k}{for} \PY{n}{k} \PY{o+ow}{in} \PY{n+nb}{range}\PY{p}{(}\PY{n+nb}{len}\PY{p}{(}\PY{n}{uninfected\PYZus{}sample}\PY{p}{)}\PY{p}{)}\PY{p}{:}
                         \PY{n}{computers}\PY{p}{[}\PY{n}{uninfected\PYZus{}sample}\PY{o}{.}\PY{n}{pop}\PY{p}{(}\PY{p}{)}\PY{p}{]} \PY{o}{=} \PY{n}{State}\PY{o}{.}\PY{n}{X\PYZus{}INF}
         
                 \PY{n}{random}\PY{o}{.}\PY{n}{shuffle}\PY{p}{(}\PY{n}{uninfected\PYZus{}sample}\PY{p}{)}
         
                 \PY{c+c1}{\PYZsh{} computers\PYZus{}df = \PYZob{}\PYZsq{}computer\PYZsq{}: [x for x in computers.keys()],}
                 \PY{c+c1}{\PYZsh{}                 \PYZsq{}state\PYZsq{}: [y for y in computers.values()]\PYZcb{}}
                 \PY{n}{df} \PY{o}{=} \PY{n}{pd}\PY{o}{.}\PY{n}{DataFrame}\PY{p}{(}\PY{n}{computers}\PY{p}{,} \PY{n}{index}\PY{o}{=}\PY{p}{[}\PY{l+s+s1}{\PYZsq{}}\PY{l+s+s1}{State}\PY{l+s+s1}{\PYZsq{}}\PY{p}{,} \PY{p}{]}\PY{p}{)}
                 \PY{n+nb}{print}\PY{p}{(}\PY{l+s+s2}{\PYZdq{}}\PY{l+s+s2}{Time: }\PY{l+s+s2}{\PYZdq{}} \PY{o}{+} \PY{n+nb}{str}\PY{p}{(}\PY{n}{time}\PY{p}{)} \PY{o}{+} \PY{l+s+s2}{\PYZdq{}}\PY{l+s+s2}{ ==\PYZgt{} [ X Infected:  }\PY{l+s+s2}{\PYZdq{}} \PY{o}{+} \PY{n+nb}{str}\PY{p}{(}\PY{n}{n\PYZus{}X}\PY{p}{)} \PY{o}{+} \PY{l+s+s2}{\PYZdq{}}\PY{l+s+s2}{ ] }\PY{l+s+se}{\PYZbs{}n}\PY{l+s+s2}{\PYZdq{}}\PY{p}{)}
                 \PY{n}{display}\PY{p}{(}\PY{n}{df}\PY{p}{)}
\end{Verbatim}


    \begin{Verbatim}[commandchars=\\\{\}]
Time: 0.0 ==> [ X Infected:  1 ] 


    \end{Verbatim}

\begin{Verbatim}[commandchars=\\\{\}]
{\color{outcolor}Out[{\color{outcolor}28}]:} <div>
         <style scoped>
             .dataframe tbody tr th:only-of-type \{
                 vertical-align: middle;
             \}
         
             .dataframe tbody tr th \{
                 vertical-align: top;
             \}
         
             .dataframe thead th \{
                 text-align: right;
             \}
         </style>
         <table border="1" class="dataframe">
           <thead>
             <tr style="text-align: right;">
               <th></th>
               <th>1</th>
               <th>2</th>
               <th>3</th>
               <th>4</th>
               <th>5</th>
               <th>6</th>
               <th>7</th>
               <th>8</th>
               <th>9</th>
               <th>10</th>
             </tr>
           </thead>
           <tbody>
             <tr>
               <th>State</th>
               <td>State.UN\_INF</td>
               <td>State.UN\_INF</td>
               <td>State.UN\_INF</td>
               <td>State.UN\_INF</td>
               <td>State.UN\_INF</td>
               <td>State.X\_INF</td>
               <td>State.UN\_INF</td>
               <td>State.UN\_INF</td>
               <td>State.UN\_INF</td>
               <td>State.X\_INF</td>
             </tr>
           </tbody>
         </table>
         </div>
\end{Verbatim}
            
    \begin{Verbatim}[commandchars=\\\{\}]
Time: 1.0 ==> [ X Infected:  2 ] 


    \end{Verbatim}

\begin{Verbatim}[commandchars=\\\{\}]
{\color{outcolor}Out[{\color{outcolor}28}]:} <div>
         <style scoped>
             .dataframe tbody tr th:only-of-type \{
                 vertical-align: middle;
             \}
         
             .dataframe tbody tr th \{
                 vertical-align: top;
             \}
         
             .dataframe thead th \{
                 text-align: right;
             \}
         </style>
         <table border="1" class="dataframe">
           <thead>
             <tr style="text-align: right;">
               <th></th>
               <th>1</th>
               <th>2</th>
               <th>3</th>
               <th>4</th>
               <th>5</th>
               <th>6</th>
               <th>7</th>
               <th>8</th>
               <th>9</th>
               <th>10</th>
             </tr>
           </thead>
           <tbody>
             <tr>
               <th>State</th>
               <td>State.UN\_INF</td>
               <td>State.UN\_INF</td>
               <td>State.X\_INF</td>
               <td>State.UN\_INF</td>
               <td>State.UN\_INF</td>
               <td>State.X\_INF</td>
               <td>State.UN\_INF</td>
               <td>State.UN\_INF</td>
               <td>State.X\_INF</td>
               <td>State.X\_INF</td>
             </tr>
           </tbody>
         </table>
         </div>
\end{Verbatim}
            
    \begin{Verbatim}[commandchars=\\\{\}]
Time: 2.0 ==> [ X Infected:  4 ] 


    \end{Verbatim}

\begin{Verbatim}[commandchars=\\\{\}]
{\color{outcolor}Out[{\color{outcolor}28}]:} <div>
         <style scoped>
             .dataframe tbody tr th:only-of-type \{
                 vertical-align: middle;
             \}
         
             .dataframe tbody tr th \{
                 vertical-align: top;
             \}
         
             .dataframe thead th \{
                 text-align: right;
             \}
         </style>
         <table border="1" class="dataframe">
           <thead>
             <tr style="text-align: right;">
               <th></th>
               <th>1</th>
               <th>2</th>
               <th>3</th>
               <th>4</th>
               <th>5</th>
               <th>6</th>
               <th>7</th>
               <th>8</th>
               <th>9</th>
               <th>10</th>
             </tr>
           </thead>
           <tbody>
             <tr>
               <th>State</th>
               <td>State.X\_INF</td>
               <td>State.UN\_INF</td>
               <td>State.X\_INF</td>
               <td>State.X\_INF</td>
               <td>State.X\_INF</td>
               <td>State.X\_INF</td>
               <td>State.UN\_INF</td>
               <td>State.X\_INF</td>
               <td>State.X\_INF</td>
               <td>State.X\_INF</td>
             </tr>
           </tbody>
         </table>
         </div>
\end{Verbatim}
            
    \begin{Verbatim}[commandchars=\\\{\}]
Time: 3.0 ==> [ X Infected:  8 ] 


    \end{Verbatim}

\begin{Verbatim}[commandchars=\\\{\}]
{\color{outcolor}Out[{\color{outcolor}28}]:} <div>
         <style scoped>
             .dataframe tbody tr th:only-of-type \{
                 vertical-align: middle;
             \}
         
             .dataframe tbody tr th \{
                 vertical-align: top;
             \}
         
             .dataframe thead th \{
                 text-align: right;
             \}
         </style>
         <table border="1" class="dataframe">
           <thead>
             <tr style="text-align: right;">
               <th></th>
               <th>1</th>
               <th>2</th>
               <th>3</th>
               <th>4</th>
               <th>5</th>
               <th>6</th>
               <th>7</th>
               <th>8</th>
               <th>9</th>
               <th>10</th>
             </tr>
           </thead>
           <tbody>
             <tr>
               <th>State</th>
               <td>State.X\_INF</td>
               <td>State.X\_INF</td>
               <td>State.X\_INF</td>
               <td>State.X\_INF</td>
               <td>State.X\_INF</td>
               <td>State.X\_INF</td>
               <td>State.X\_INF</td>
               <td>State.X\_INF</td>
               <td>State.X\_INF</td>
               <td>State.X\_INF</td>
             </tr>
           </tbody>
         </table>
         </div>
\end{Verbatim}
            
    \hypertarget{question-7.}{%
\subsubsection{Question 7.}\label{question-7.}}

    \hypertarget{xml-bomb}{%
\paragraph{XML Bomb}\label{xml-bomb}}

A form of Denial-of-Service attack that impacts the availability of a
website. XML, eXtensible Markup Language, is a markup language that was
designed and is popular for storing and transferring structured data
{[}1{]}. Data files can be checked by a parsing library before being
processed. XML schemas and Document Type Definitions (DTDs) files are
used to then validate and compare rules for the type of data that
appears in the XML file. Inline DTDs can be abused when the parsing
library is not configured properly, causing what is known as an XML bomb
or Entity Expansion XML bomb {[}2{]}. A famous XML bomb example includes
the ``Billions Laugh Effect.'' This type of attack can list entity
definitions in an XML file that is heavily nested, leading to parsing
that can lead to extremly large files. The large size of these parsed
XML files can cause a server to crash.

\hypertarget{bluesmack}{%
\paragraph{BlueSmack}\label{bluesmack}}

A form of Denial-of-Service attack that specifically targets
Bluetooth-enabled devices and the Bluetooth protocol. It is a similar
technique to the `Ping of Death' attack of Windows 95, where a l2ping
data packet (roughly 600 bytes) is transmitted to a Bluetooth device
bigger than the maximum packet size {[}3{]}, {[}4{]}. This causes the
device to become unusable, and can also affect the device by draining
its battery {[}4{]}. The l2ping packet is part of the standard
distribution of the BlueZ utility package that ships with Linux {[}3{]}.

\hypertarget{mydoom}{%
\paragraph{Mydoom}\label{mydoom}}

Mydoom (aka W32.MyDoom@mm) is a variant of computer worm that was spread
via email which affected Microsoft Windows systems discovered in 2004.
It infected systems by copying itself to P2P KaZaA client shared
directories and its payload contained a backdoor Remote Access component
on TCP port 3127 {[}5{]}. From here, the worm was able to extract email
addresses from the system and then use SMTP methods to send itself as an
attachment from the host email server {[}5{]}. The original variant of
Mydoom is the fastest growing spam email worm known so far and has also
been shown to avoid sending itself to certain domain addresses such as
Rutgers, MIT, Stanford, UC Berkeley, Microsoft and Symantec {[}6{]}.
Other variants of the worm were later designed to use infected hosts as
zombies in a Distributed Denial-of-Service attack directed at the SCO
Group and Microsoft {[}6{]}.

\hypertarget{torpig}{%
\paragraph{Torpig}\label{torpig}}

\hypertarget{references}{%
\paragraph{References:}\label{references}}

{[}1{]} w3schools, ``Introduction to XML,'' \emph{w3schools.com}.
{[}Online{]}. Available: https://www.w3schools.com/xml/xml\_whatis.asp
{[}Accessed: Sep.~23, 2018{]}.

{[}2{]} D. Jovanoski, ``XML vulnerabilities,'' May 6, 2013.
{[}Online{]}. Available:
https://resources.infosecinstitute.com/xml-vulnerabilities/ {[}Accessed:
Sep.~23, 2018{]}.

{[}3{]} trinite.suff, ``BlueSmack,'' \emph{tifinite.org}. {[}Online{]}
Available: https://trifinite.org/trifinite\_stuff\_bluesmack.html
{[}Accessed: Sep.~23, 2018{]}

{[}4{]} S.P. Oriyano, \emph{Kali Linux Wireless Penetration Testing
Cookbook}. Birmingham: Packt Publishing, 2017.

{[}5{]} McAfee, ``Virus Profile: W32/Mydoom@MM,'' \emph{McAfee},
Jan.~02, 2004. {[}Online{]}. Available:
https://home.mcafee.com/virusinfo/virusprofile.aspx?key=100983
{[}Accessed: Sep.~23, 2018{]}.

{[}6{]} Newsweek Staff, ``More Doom?'' \emph{Newsweek}, Feb.~2, 2004.
{[}Online{]}. Available: https://www.newsweek.com/more-doom-131157
{[}Accessed: Sep.~23, 2018{]}.

{[}7{]} B. Stone-Gross, M. Cova, B. Gilbert, R. Kemmerer, C. Kruegel,
and G. Vigna, "Analysis of a Botnet Takeover,' \emph{IEEE Security \&
Privacy}, vol.~9, no. 1, Jan.-Feb, pp.~64-72, 2010. Available: IEEE
Xplore Digital Library, https://ieeexplore.ieee.org/document/5560627/.
{[}Accessed September 23, 2018{]}.

{[}8{]} Carnegie Mellon University, ``Torpig,'' \emph{web.archive.org},
May 19, 2015. {[}Online{]}. Available:
https://web.archive.org/web/20150519174934/http://www.cmu.edu/iso/aware/be-aware/torpig.html
{[}Accessed: Sep.~23, 2018{]}.

{[}9{]} R. Naraine, ``Botnet hijack: Inside the Torpig malware
operation,'' \emph{zdnet.com}, May 4, 2009. {[}Online{]} Available:
https://www.zdnet.com/article/botnet-hijack-inside-the-torpig-malware-operation/
{[}Accessed: Sep.~23, 2018{]}.

    \hypertarget{question-8.}{%
\subsubsection{Question 8.}\label{question-8.}}

    A Browser Helper Object (BHO) is a plugin that is used in Microsoft's
Internet Explorer (IE) that added additional functionality, such as a
toolbar, to the browsers Document Object Model (DOM) {[}2{]}. The
objects are commonly DLL, DAT, or EXE files that are installed
mistakenly by users or often as part of bundled software.

It is for its designed and intended purpose that BHOs were used
maliciously in an attack commonly called Browser Hijack Objects, which
is a form of adware. This attack affects the integrity and availability
of the system {[}1{]}. In this malicious infection, the plugin that is
installed is often used to provide advertisements in a toolbar that can
provide financial incentives for the attacker when clicked on. Through
the use of BHOs, the attacker can automate the the clicks for financial
gain every time a victim uses their browser.

\hypertarget{references}{%
\paragraph{References:}\label{references}}

{[}1{]} W. Stallings, \emph{Computer Security: Principles and Practice,
Global Edition, 4th Edition}, New York, NY: Pearson, 2018.
{[}VitalSource Bookshelf version{]}. Available:
https://www.vitalsource.com/en-uk/products/computer-security-principles-and-practice-global-william-stallings-v9781292220635.
{[}Accessed: Sep.~23, 2018{]}.

{[}2{]} Malwarebytes Labs, ``Browser Hijack Objects (BHOs),''
\emph{Malwarebytes Labs}, Jun.~9, 2016. {[}Online{]}. Available:
https://blog.malwarebytes.com/threats/browser-hijack-objects-bhos/
{[}Accessed: Sep.~23, 2018{]}.


    % Add a bibliography block to the postdoc
    
    
    
    \end{document}
